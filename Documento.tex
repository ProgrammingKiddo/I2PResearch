\documentclass{article}
%\usepackage[utf8]{inputenc}
\usepackage{lmodern,textcomp}
\usepackage{hyperref}
\usepackage{graphicx}
\usepackage{float}
%\usepackage{listings}


\title{Red Anónima I2P}
\author{Borja Fernández Merchán \\ Andrés Gomar Pérez \\ Carlos Rodrigo Sanabria Flores \\ Sergio Ruiz Pino
 \\ Iván Piña Arévalo}
\date{\today}

\begin{document}
\maketitle

\section{Introducción}
En este trabajo realizaremos un estudio sobre la tecnología I2P. En el presente Internet forma parte de la vida 
de millones de personas y afortunadamente cada vez somos más conscientes de la importancia de los datos. El bien 
más preciado para muchas empresas es precisamente los datos de sus potenciales clientes. Además, con casos tan 
flagrantes como Cambridge Analytics o las fugas de información a las que desafortunadamente estamos acostumbrados. 

Tal vez un sentimiento similar surgió en la persona que se esconde bajo el pseudónimo ‘jrandom’. Bajo este 
pseudónimo se oculta la mente que esta al cargo del proyecto I2P. Pero antes pongamos un ejemplo que puede resultar 
esclarecedor para entender.

Supongamos a un activista, dicho activista está en contra de las medidas que se están tomando en su país contra 
la fauna y flora que habita allí. Sin embargo, debido a las relaciones diplomáticas con los países de su alrededor
y a la represión mediática, dicha información no traspasa las fronteras nacionales, impidiendo que se sepa las desastrosas 
decisiones que se están llevando a cabo. En este caso lo más importante para el activista no es la información en si misma 
ya que la conocen centenares o miles de personas. Lo más importante es que no se conozca su verdadera identidad ya que 
automáticamente seria localizado y según la normativa del país podría incluso ser ejecutado. 

En respuesta a este tipo de situaciones surge el concepto de redes anónimas. En este tipo de redes el objetivo principal es 
el anonimato. El objetivo principal de Internet cuando surgió fue la transmisión de información, no la privacidad de esta. La 
capa que subyace a este tipo de tecnologías es el enrutamiento basado en capas de cebolla. De esta manera conseguimos que cada 
nodo conozca únicamente su nodo anterior y siguiente en lugar de los nodos origen y final. Otro aspecto fundamental para conseguir 
la deseada privacidad consiste en el desacomplamiento del emisor y su mensaje. De esta forma conseguimos ganar anonimato en las 
transmisiones. 

Cabe destacar una particularidad de la red I2P. Se puede observar como si fuera una intranet. Esta característica en si misma se podría ver 
como una mejora de seguridad ya que limitamos el número de tecnologías y dispositos capaces de acceder desde el exterior. Al limitar las tecnologías 
empleadas reducimos el número de posibles amenazas, en otras palabras, podemos prestar más atención a la tipología de los ataques. 

Por los motivos mencionados anteriormente y el momento actual en el que vivimos, realizar un estudio acerca de una tecnología que 
pone su enfoque en la privacidad del usuario resulta esperanzador. Por supuesto es un arma de doble filo ya que puede ser empleada con fines 
perversos, pero no deja de ser una herramienta, todo depende del uso que queramos hacer de ella.


\section{Funcionamiento}
\section{Utilidades}
\section{Diferencia respecto a otras tecnologías}
\subsection{Ventajas}
\subsection{Inconvenientes}
\section{I2P sobre linux}
\section{Bibliografía}
\begin{itemize}
    \item %http://scielo.sld.cu/scielo.php?script=sci\_arttext&pid=S2218-36202017000200014
\end{itemize}
\end{document}