\documentclass{article}
%\usepackage[utf8]{inputenc}
\usepackage{lmodern,textcomp}
\usepackage{hyperref}
\usepackage{graphicx}
\usepackage{float}
%\usepackage{listings}


\title{Red Anónima I2P}
\author{Borja Fernández Merchán \\ Andrés Gomar Pérez \\ Carlos Rodrigo Sanabria Flores \\ Sergio Ruiz Pino
 \\ Iván Piña Arévalo}
\date{\today}

\begin{document}
\maketitle

\section{Introducción}
En este trabajo realizaremos un estudio sobre la tecnología I2P. En el presente Internet forma parte de la vida 
de millones de personas y afortunadamente cada vez somos más conscientes de la importancia de los datos. El bien 
más preciado para muchas empresas es precisamente los datos de sus potenciales clientes. Además, con casos tan 
flagrantes como Cambridge Analytics o las fugas de información a las que desafortunadamente estamos acostumbrados. 

Tal vez un sentimiento similar surgió en la persona que se esconde bajo el pseudónimo ‘jrandom’. Bajo este 
pseudónimo se oculta la mente que esta al cargo del proyecto I2P. Pero antes pongamos un ejemplo que puede resultar 
esclarecedor para entender.

Supongamos a un activista, dicho activista está en contra de las medidas que se están tomando en su país contra 
la fauna y flora que habita allí. Sin embargo, debido a las relaciones diplomáticas con los países de su alrededor
y a la represión mediática, dicha información no traspasa las fronteras nacionales, impidiendo que se sepa las desastrosas 
decisiones que se están llevando a cabo. En este caso lo más importante para el activista no es la información en si misma 
ya que la conocen centenares o miles de personas. Lo más importante es que no se conozca su verdadera identidad ya que 
automáticamente seria localizado y según la normativa del país podría incluso ser ejecutado. 

En respuesta a este tipo de situaciones surge el concepto de redes anónimas. En este tipo de redes el objetivo principal es 
el anonimato. El objetivo principal de Internet cuando surgió fue la transmisión de información, no la privacidad de esta. La 
capa que subyace a este tipo de tecnologías es el enrutamiento basado en capas de cebolla. De esta manera conseguimos que cada 
nodo conozca únicamente su nodo anterior y siguiente en lugar de los nodos origen y final. Otro aspecto fundamental para conseguir 
la deseada privacidad consiste en el desacomplamiento del emisor y su mensaje. De esta forma conseguimos ganar anonimato en las 
transmisiones. 

Cabe destacar una particularidad de la red I2P. Se puede observar como si fuera una intranet. Esta característica en si misma se podría ver 
como una mejora de seguridad ya que limitamos el número de tecnologías y dispositos capaces de acceder desde el exterior. Al limitar las tecnologías 
empleadas reducimos el número de posibles amenazas, en otras palabras, podemos prestar más atención a la tipología de los ataques. 

Por los motivos mencionados anteriormente y el momento actual en el que vivimos, realizar un estudio acerca de una tecnología que 
pone su enfoque en la privacidad del usuario resulta esperanzador. Por supuesto es un arma de doble filo ya que puede ser empleada con fines 
perversos, pero no deja de ser una herramienta, todo depende del uso que queramos hacer de ella.

\section{Casos de estudio en el que se aplica la plataforma/tecnologia I2P}

En el caso de la web, el ideal no cambia, es decir, lo que se busca es usar
comunicaciones cifradas para mantener la privacidad de las páginas que visitan 
los usuarios mediante saltos entre los nodos de la red.

Hay que destacar que en cuanto a las diferentes tecnologías actuales, si lo que
queremos es conectarnos a internet a modo de proxy,  Tor es más eficiente que I2P,
desventaja de ésta última que la hace más privada que las demás tecnologías al tener
menos usuarios siendo mas resistente a ataques hacia la privacidad.[1]

Hay actualmente diversos estudios de investigación referente a la confiabilidad y 
seguridad de las comunicaciones que se producen en el día a día.
Existe una gran preocupación que concierne a los programas de vigilancia en la web
aplicados en plataformas como Facebook, amazon, Google …
Esto ha llevado a un colectivo a realización investigaciones para ver de que manera
se puede proteger la integridad de los datos. El objetivo de estas investigaciones a
llevado a obtener soluciones que nos permitan transferir datos sin que haya terceros
en la red que puedan interceptar éstos.
Una solución propuesta es usar tecnología I2P para asegurar las comunicaciones en la
red. Enviamos paquetes desde uno de los túneles de salida hacia un túnel de entrada 
impidiendo que nadie puede interceptarlos salvo el receptor del paquete.[2]

Un ejemplo de implementación de I2P es el aseguramiento de privacidad en las 
comunicaciones sobre internet.
La aparición de nuevas tecnologías ha conllevado la preocupación de colectivos
centrados en la privacidad de los datos que se tratan en la red en un ámbito público
o privado.

El proyecto I2P se encuentra en auge y continua evolución para garantizarnos el 100%
de seguridad e integridad de los datos y paquetes que se manejan en la red. Éste se 
es resistente a ataques masivos posibilitando la ejecución de varias aplicaciones a 
la vez ya que consta de varios modelos a nivel de seguridad y rendimiento para cada 
usuario que aplican esta tecnología. Las aplicaciones que se basan en esta tecnología
se usan básicamente en internet, en la sección de navegación web anonimo por ejemplo.[3]


Referencias
[1] https://www.genbeta.com/actualidad/i2p-la-nueva-generacion-de-la-deep-web
[2] http://scielo.sld.cu/scielo.php?script=sci_arttext&pid=S2218-36202017000200014
[3] http://repositorio.ug.edu.ec/bitstream/redug/17047/1/UG-FCMF-B-CINT-PTG-N.106.pdf


\section{Funcionamiento}
\section{Utilidades}
\section{Diferencia respecto a otras tecnologías}
\subsection{Ventajas}
\subsection{Inconvenientes}
\section{I2P sobre linux}
\section{Bibliografía}
\begin{itemize}
    \item %http://scielo.sld.cu/scielo.php?script=sci\_arttext&pid=S2218-36202017000200014
\end{itemize}
\end{document}
